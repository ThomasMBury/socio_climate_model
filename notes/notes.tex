
%----------------------------------------------------------------------------------------
%	PACKAGES AND OTHER DOCUMENT CONFIGURATIONS
%----------------------------------------------------------------------------------------

\documentclass[10pt]{article} % A4 paper and 11pt font size

\usepackage[T1]{fontenc} % Use 8-bit encoding that has 256 glyphs
\usepackage[english]{babel} % English language/hyphenation
\usepackage{amsmath,amsfonts,amsthm} % Math packages
\usepackage{enumitem} % Package for lists
\usepackage{graphicx} % Package for figures
\usepackage{changepage} % Allows for change of margin during paragraph
\usepackage[total={6.7in,8in}]{geometry} % Span of text over the page
\usepackage{verbatim}
\usepackage{float}
\usepackage{bm}		% For bold greek letters
\usepackage{graphicx}
\usepackage{caption}
\usepackage{subcaption}
\usepackage{array}

\usepackage{fancyhdr} % Custom headers and footers
\pagestyle{headings} % Makes all pages in the document conform to the custom headers and footers

\setlength\parindent{0pt} % Removes all indentation from paragraphs

\renewcommand{\vec}[1]{\bm{#1}} % Use bold vectors

\DeclareMathOperator{\Tr}{Tr} % For trace and determinant text in equations
\DeclareMathOperator{\Det}{Det}
\DeclareMathOperator{\Var}{Var}
\DeclareMathOperator{\Cov}{Cov}





\begin{document}


\section{Contrasting with Beckage}


\begin{table}[H]
\centering
\begin{tabular}{m{25em}|m{25em}}
\textbf{Beckage} & \textbf{Us}\\
\hline\\
Extreme weather patterns influence human behaviour. So only current events. & Human behaviour influenced by projected temperature at time $t_f$. \\
& \\
Individual level psychology scaled up to population level (theory of planned behaviour) &  Social dynamics between individuals (imitation dynamics)\\
& \\
Uses discrete climate events and sensing/forgetting process & Not based on discrete events - just a value of temperature at some projected time\\
& \\
Both use dynamic feedback between climate and human perception and vice-versa & \\
& \\
Three different forms for response to frequency of extreme events (logistic, linear, cubic) & I think sigmoidal with threshold around the 2 degree mark is most relevant for us - can contrast this to linear. Cubic and logistic both give quite extreme results for our model (6 degrees, 1.5 degrees respectively).  \\
& \\
Examine cumulative and non-cumulative mitigative responses & One general mitigative response - more responses could be an interesting extension. \\
& \\
Entire population adopts the same response & Individuals may adopt one of two strategies \\
& \\
Projected temperature range for 2100 (3.4-5.9) degC - very high - note that IPCC range is (1.5-4.5) & Projected temp (1.8-4.5 degC) for near baseline values of $\kappa$ and $\delta$. This could go much higher for lower kappa, higher delta. \\
& \\
Impose limits on annual shifts of carbon flux and a minimum level of anthropogenic emissions & We do not (simpler model) \\
& \\
Demonstrate that model sensitivity to climate parameters is similar to sensitivity to social parameters (find similar temperature projection ranges) & We could do this.\\
& \\
Components with largest increase on temperature: social norms, perceived behavioural control (related to our cost of mitigation) , cumulative vs. non-cumulative mitigation strategies, functional form of response to temp. & Social norms (but dynamic here), functional form of temp. response, social learning rate, distance in time looking ahead. \\
& \\

\end{tabular}
\end{table}














% ---------------------------
% Bibliography
%---------------------------

\pagebreak

\bibliographystyle{unsrt}
\bibliography{../../bibliographies/critical_transitions.bib}



\end{document}


